\section{Introduction}

Formal verification is a subject that constantly attracts attention of the research community. Static type systems are considered to be a lightweight verification method but they can be very powerful and precise as well. Dependent type systems in languages like Agda~\cite{Nor07}, Idris~\cite{Bra13} or Coq~\cite{coq} can be succesfully applied in practical verification tasks but they are not yet as widely used as they could potentially be. This paper contributes to changing that.

\subsection{Motivation}

Two things have motivated me to write this paper. Firstly, while there are many tutorials on dependently typed programming and basics of verification, I could find little material demonstrating how to put verification to practical use. A must-read introductory paper ``Why Dependent Types Matter'' by Altenkirch, McKinna and McBride \cite{AltMcBMcK05}, which demonstrates how to use dependent types to prove correctness of merge sort algorithm, actually elides many proof details that are required in a real-world application. I want to fill in that missing gap by writing a tutorial that picks up where other tutorials have ended.

My second motivation comes from reading Okasaki's classical ``Purely Functional Data Structures''~\cite{Oka99}. Despite book's title many presented implementations are not purely functional as they make use of impure exceptions to handle corner cases (eg. taking head of an empty list). I realized that using dependent types allows to do better and it is instructive to build a provably correct purely functional data structure on top of Okasaki's presentation.

In the end this paper is both a tutorial and a case study of weight biased leftist heap implemented in dependently typed setting. My goal is to teach the reader how to build complex proofs from simple ones. As a result the reader will be able to verify that operations on a data structure maintain required invariants. Acquired knowledge will allow to understand more advanced verification techniques, eg. equational reasoning provided by Agda's standard library or tactics system found in Idris \cite{Bra13} and Coq \cite{coq}.

\subsection{Companion code}

This tutorial comes with a standalone companion code written in Agda 2.3.4\footnote{\url{http://ics.p.lodz.pl/~stolarek/_media/pl:research:dep-typed-wbl-heaps.tar.gz}}. I assume the reader is reading companion code along with the paper. Due to space limitations I elide some proofs that are detailed in the code using Notes convention addapted from GHC project~\cite{MarPey12}.

``Living'' version of companion code is available at GitHub\footnote{\url{https://github.com/jstolarek/dep-typed-wbl-heaps}} and it may receive updates after the paper is published.

\subsection{Assumptions}

I assume that reader has basic understanding of Agda, some elementary definitions and proofs. In particular I assume the reader is familiar with definition of natural numbers (\texttt{Nat}s) and their addition (\texttt{+}) as well as proofs of basic properties of addition like associativity, commutativity or 0 as right identity ($a + 0 ≡ a$). Reader should also understand \texttt{refl} with its basic properties (symmetry, congruence, transitivity and substitution), know the concept of ``data as evidence'' and other ideas presented in ``Why Dependent Types Matter'' \cite{AltMcBMcK05} as I will build upon them. All of these are implemented in the \texttt{Basics} module in the companion code. Module \texttt{Basics.Reasoning} reviews in detail the above-mentioned proofs.

\subsection{Notation and conventions}

In the rest of the paper I will denote heaps using \texttt{typewriter font} and their ranks using an \textit{italic type}. The description of merge algorithm will mention heaps \texttt{h1} and \texttt{h2} with ranks \textit{h1} and \textit{h2} respectively, their left children (\texttt{l1} in \texttt{h1} and \texttt{l2} in \texttt{h2}) and right children (\texttt{r1} in \texttt{h1} and \texttt{r2} in \texttt{h2}) with \texttt{p1} and \texttt{p2} as the priorities of root elements in \texttt{h1} and \texttt{h2} respectively. In the text I will use $\oplus$ to denote heap merging operation. So \texttt{h1}$\oplus$\texttt{h2} will be a heap created by merging \texttt{h1} with \texttt{h2}, while \textit{h1}$\oplus$\textit{h2} will be the rank of the merged heap.

I will represent priority using natural numbers with lower number meaning higher priority. This means that 0 will be the highest priority, while the lowest priority will be unbounded. This also means that if \texttt{p1 > p2} holds as a relation on natural numbers then \texttt{p2} is higher priority than \texttt{p1}.

In the text I will use numerals to represent \texttt{Nat}s but the code uses encoding based on \texttt{zero} and \texttt{suc}. Thus 2 in the text will correspond to \texttt{suc (suc zero)} in the source code.

I will use \texttt{\hilight{\{ \}?}} in code listings to represent Agda holes.

Remember that any sequence of Unicode characters is a valid identifier in Agda. Thus \texttt{l≥r} is an identifier, while \texttt{l ≥ r} is application of \texttt{≥} operator to \texttt{l} and \texttt{r} operands.

\subsection{Contributions}

This paper contributes the following:

\begin{itemize}
 \item Section~\ref{sec:no-proofs} presents unverified implementation and the problem of partiality of functions operating on a weight biased leftist heap. While the problem in general is well-known the solution to this particular case can be combined with verification of one of data structure's invariants. This is done in Section~\ref{sec:rank-property}.
 \item Section~\ref{sec:eq-proofs-using-trans} outlines a technique for constructing equality proofs using transitivity of propositional equality. This simple, standalone technique provides ground for understanding verification mechanisms used in Agda's standard library.
 \item Section~\ref{sec:single-pass-merge-proof-by-comp} uses the technique introduced in Section~\ref{sec:eq-proofs-using-trans} to prove code obtained by inlining one function into another. This shows how programs created from small, verified components can be proved correct by composing proofs of these components.
 \item Section~\ref{sec:priority-invariant} contains a case study of how a proof of data structure invariant influences the design of an API. This is demonstrated on the example of priority invariant proof and its influence on designing insertion of a new element into a heap.
\end{itemize}
